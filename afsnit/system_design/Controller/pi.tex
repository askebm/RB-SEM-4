\documentclass[../../../Main]{subfiles}
\begin{document}
\label{sec:PI}
In this section the controllers for the top and bottom frames of the pan-tilt system are designed and tuned, so they conform to the system requirements in the problem statement see table \ref{tab:performance_specifications}.

\subsubsection{Tuning of Top Frame}
\label{sec:top_frame_tuning}
The top frame is nonlinear, as can be seen in equation \eqref{eq:new_mech_equation_with_corners}, which includes a sine term.

To design a controller, for the top frame, the system is linearized in four points. This is done because the actual system step response looks more like a sawtooth pattern, than an actual sine pattern. See journal figure \ref{fig:jour_step_bot}.

By doing this, it is possible to have linear behavior for the system in general.
Because $-90^o$ and  $+90^o$ have the same behavior in this system, only 3 linearizations are made.

\paragraph{Linearization in $0^\circ$}
\label{sec:linearize}
The system is linearized in $0^\circ$. Which gives the following A, B and C matrices.


\begin{equation}
      \begin{split}
      \label{eq:zero_linearized}
      \frac{d}{dt}
    \begin{bmatrix}
        \theta_t \\
        \dot \theta_t \\
        i_t
    \end{bmatrix}
    =&
    \begin{bmatrix}0 & 1.0 & 0\\ -10.8 & -75.8 & 212.0\\ 0 & -260.0 & -1.24\cdot10^{4}\end{bmatrix}
    \begin{bmatrix}
        \theta_t \\
        \dot \theta_t \\
        i_t \\
    \end{bmatrix}
    +
    \begin{bmatrix}
        0 \\
        0 \\
	\frac{1}{3.85\cdot10^{-4}}\\
    \end{bmatrix}
    V_t
\\
      y =&
    \begin{bmatrix}
        1 & 0 & 0
    \end{bmatrix}
    \begin{bmatrix}
        \theta_t \\
        \dot \theta_t\\
        i_t\\
    \end{bmatrix}
    \end{split}
\end{equation}
\todo{reg/punktum hele fil}
\begin{figure}[H]
\centering
\def\svgwidth{\textwidth}
\import{\main/afsnit/system_design/Controller/img/PI/}{rlocus_pure.pdf_tex}
\caption{Root locus of openloop plant without controller}
\label{fig:rlocus_pure}
\end{figure}

Based on the root locus, the system can be controlled and be stable using only a P-gain.
However this will result in a steady-state error. For this reason a PI-controller is tuned.
A PI-controller is given by transfer function \eqref{equ:transfPI}

\begin{align}
	K_p \frac{s+\frac{1}{T_i}}{s}
	\label{equ:transfPI}
\end{align}
From equation \eqref{equ:transfPI} it is seen, that a zero is added to the system and is placed in $-\frac{1}{T_i} $
The controller must be tuned to comply with the system requirements set forth in the problem statement. See table \ref{tab:performance_specifications}. For this to apply, the controlled system must be overdamped with real poles. And these poles must be located further to the left than $\sigma$ where $\sigma$ is given by is given by
\begin{align}
	\sigma &\ge \frac{4.6}{t_r}\\
    t_r &= 1 \si{s}\\
	\sigma &\ge 4.6
\end{align}

With a $T_i = \frac{1}{0.035}$ the root locus in figure \ref{fig:rlocus_int} is generated.

\begin{figure}[H]
\centering
\def\svgwidth{\textwidth}
\import{\main/afsnit/system_design/Controller/img/PI/}{rlocus_int.pdf_tex}
\caption{Root locus of openloop plant with integral regulator}
\label{fig:rlocus_int}
\end{figure}

With a gain $K_p = 3.1$ a dominant pole is placed at $-5$. And the pole in 0, would approach the zero. Which should make this system mimic a first order system. The system controlled by a controller with gains $T_i = \frac{1}{0.035}$ and $K_p = 3.1$ yields a system with the step response in figure \ref{fig:step_0_first}

\begin{figure}[H]
\centering
\def\svgwidth{\textwidth}
\import{\main/afsnit/system_design/Controller/img/PI/}{step_0_first.pdf_tex}
\caption{Step response of system with the tuned PI-regulator. The dotted vertical line is 1s}
\label{fig:step_0_first}
\end{figure}

\begin{table}[h]
	 \centering
	 \begin{tabular}{ll}
		\textbf{Overshoot} & 0\%\\
		 \textbf{Risetime} & 0.4798s\\
		 \textbf{Settling Time} & 2.1619s
	 \end{tabular}
	 \caption{Performance characteristics of PI-controlled system in figure \ref{fig:step_0_first}.}
	 \label{tab:performance_0_1}
\end{table}

In figure \ref{fig:step_0_first} it can be seen that the rise time is faster than the expected 1 s, and the reference is not reached. Therefore the $K_p$ and $T_i$ are tuned to better match the requirements. $K_p = 1.6$ and $T_i=0.2$ are found to give the closest to expected response. The step response can be seen in figure \ref{fig:step_0_second}.

\begin{figure}[H]
\centering
\def\svgwidth{\textwidth}
\import{\main/afsnit/system_design/Controller/img/PI/}{step_0_second.pdf_tex}
\caption{Step response and bode plot of system with the tuned PI-regulator}
\label{fig:step_0_second}
\end{figure}


\begin{table}[h]
	 \centering
	 \begin{tabular}{ll}
		 \textbf{Overshoot} & 1.7522\%\\
		 \textbf{Risetime}  & 0.8318s\\
		 \textbf{Settling Time} & 1.3199s\\
		 \textbf{Gain Margin} & 74.4dB\\
		 \textbf{Phase Margin} & 86.7$^\circ$
	 \end{tabular}
	 \caption{Performance and stability characteristics of PI-controlled system in figure \ref{fig:step_0_second}.}
	 \label{tab:performance_0_2}
\end{table}



The regulated system has a slight overshoot, but has a rise time of 1 second as per the system requirements. The overshoot is the lowest possible, if the reference is to be reached. Furthermore the system takes 100 seconds to reach its final value.

\paragraph{Linearization in $90^\circ$ and $180^\circ$}
In this section the linearized system is tuned in the same manner as it is in section \ref{sec:linearize}. But not all steps are shown, as the approach is the same. The tuned values and the final step responses will be shown and commented.

Linearization in $90^\circ$ gives A, B and C matrices shown in equation \ref{eq:90_linearized}.

\begin{equation}
      \label{eq:90_linearized}
      \begin{split}
      \frac{d}{dt}
    \begin{bmatrix}
        \theta_t \\
        \dot \theta_t \\
        i_t
    \end{bmatrix}
    &=
    \begin{bmatrix}0 & 1.0 & 0\\ 0  & -75.8 & 212.0\\ 0 & -260.0 & -1.24 \cdot 10^4 \end{bmatrix}
    \begin{bmatrix}
        \theta_t \\
        \dot \theta_t \\
        i_t \\
    \end{bmatrix}
    +
    \begin{bmatrix}
     0\\ -1.57\\ 2600.0
    \end{bmatrix}
    V_t \\
		y &=
    \begin{bmatrix}
        1 & 0 & 0
    \end{bmatrix}
    \begin{bmatrix}
        \theta_t \\
        \dot \theta_t\\
        i_t
    \end{bmatrix}
    \end{split}
\end{equation}

Figure \ref{fig:step_0_second_90} shows the tuned step response of the closed loop control system and plant.
It is tuned to $K_p = 1.39$ and $T_i = \frac{1}{0.006}$
\begin{figure}[H]
\centering
\def\svgwidth{\textwidth}
\import{\main/afsnit/system_design/Controller/img/PI/}{step_0_second_90.pdf_tex}
\caption{Step response and stability  of system with the tuned PI-regulator.}
\label{fig:step_0_second_90}
\end{figure}


\begin{table}[h]
	 \centering
	 \begin{tabular}{ll}
		 \textbf{Overshoot} & 2.5603\%\\
		 \textbf{Risetime}  & 0.9890s\\
		 \textbf{Settling Time} & 8.1063s\\
		 \textbf{Gain Margin} & 75.6dB\\
		 \textbf{Phase Margin} & 86.8$^\circ$
	 \end{tabular}
	 \caption{Performance characteristics and stabillity margins of PI-controlled system in figure \ref{fig:step_0_second_90}.}
	 \label{tab:performance_0_3}a
\end{table}

Linearization in $180^\circ$ gives A, B and C matrices shown in equation \ref{eq:180_linearized}.

\begin{equation}
      \label{eq:180_linearized}
      \begin{split}
      \frac{d}{dt}
    \begin{bmatrix}
        \theta_t \\
        \dot \theta_t \\
        i_t
    \end{bmatrix}
    =&
    \begin{bmatrix}0 & 1.0 & 0\\ 10.8 & -75.8 & 212.0\\ 0 & -260.0 & -1.24\cdot10^4 \end{bmatrix}
    \begin{bmatrix}
        \theta_t \\
        \dot \theta_t \\
        i_t \\
    \end{bmatrix}
    +
    \begin{bmatrix}
    0\\ -1.0\\ 2600.0
    \end{bmatrix}
    V_t
\\
      y =&
    \begin{bmatrix}
        1 & 0 & 0
    \end{bmatrix}
    \begin{bmatrix}
        \theta_t \\
        \dot \theta_t\\
        i_t\\
    \end{bmatrix}
    \end{split}
\end{equation}


Figure \ref{fig:step_0_second_180} shows the tuned step response of the closed loop control system and plant.
it is tuned to $K_p = 1.3$ and $T_i = \frac{1}{0.2}$
\begin{figure}[H]
\centering
\def\svgwidth{\textwidth}
\import{\main/afsnit/system_design/Controller/img/PI/}{step_0_second_180.pdf_tex}
\caption{Step response and bodeplot of system with the tuned PI-regulator.}
\label{fig:step_0_second_180}
\end{figure}

\begin{table}[h]
	 \centering
	 \begin{tabular}{ll}
		 \textbf{Overshoot} & 13.5409\%\\
		 \textbf{Overshoot} & 0.8075s\\
		 \textbf{Settling Time} & 10.8682s\\
		 \textbf{Positive Gain Margin} & 76.2dB\\
		 \textbf{Negative Gain Margin} & -22.9dB\\
		 \textbf{Phase Margin} & 78.5$^\circ$
	 \end{tabular}
	 \caption{Performance characteristics and stability margins of PI-controlled system in figure \ref{fig:step_0_second_180}.}
	 \label{tab:performance_0_4}
\end{table}

The system responses are not exactly similar, and especially the system when linearized in $180^\circ$ does not look like the other step responeses. Also the system is less stable, as it has a smaller gain margin, than the other linearizations. In figure \ref{fig:comp}, all three responses are shown in the same plot, for comparison. Furthermore none of the tuned controllers reach all of the requirements, however they reach the best compromise for the system.

\begin{figure}[H]
\centering
\def\svgwidth{\textwidth}
\import{\main/afsnit/system_design/Controller/img/PI/}{comp.pdf_tex}
\caption{Step response of all three PI-controlled systems}
\label{fig:comp}
\end{figure}

\subsubsection{Tuning of Bottom Frame}

In contrast to the top frame, the bottom frame is linear assuming the top frame is stationary. It therefore is not required to linearize it. 
As the top frame can move, it influences the moment of inertia of the bottom frame as a function of the top frames angle in relation to the bottom frame, as described in section \ref{sec:moment_botom}.
In the following subsection it is therefore investigated whether or not the top frame's position actually has an influence.
This is done by tuning the bottom frame controller for when the top frame is at respectively $90^\circ$ and $0^\circ$.
The state space model of the bottom frame, when the topframe is at $0^\circ$, is found in equation \eqref{eq:bottom_state_eq}

\begin{equation}
      \label{eq:bottom_state_eq}
      \begin{split}
      \frac{d}{dt}
    \begin{bmatrix}
        \theta_t \\
        \dot \theta_t \\
        i_t
    \end{bmatrix}
    =&
    \begin{bmatrix}0 & 1.0 & 0\\ 0 & -35.0 & 110.0\\ 0 & -9099.0 & -1.24\cdot10^4\end{bmatrix}
    \begin{bmatrix}
        \theta_t \\
        \dot \theta_t \\
        i_t \\
    \end{bmatrix}
    +
    \begin{bmatrix}
      0\\ 0\\ 2600
    \end{bmatrix}
    V_t
\\
      y =&
    \begin{bmatrix}
        1 & 0 & 0
    \end{bmatrix}
    \begin{bmatrix}
        \theta_t \\
        \dot \theta_t\\
        i_t\\
    \end{bmatrix}
    \end{split}
\end{equation}


\begin{figure}[H]
\centering
\def\svgwidth{\textwidth}
\import{\main/afsnit/system_design/Controller/img/PI/}{rlocus_pure_bot.pdf_tex}
\caption{Root locus of openloop plant without regulator}
\label{fig:rlocus_pure_bot}
\end{figure}

In figure \ref{fig:rlocus_pure_bot} a root locus of the openloop system is seen. As in section \ref{sec:top_frame_tuning} the system can be controlled only by a P-gain. 
But it will likely result in a steady state error. 
Therefore the system is tuned with a PI-controller. In figure \ref{fig:rlocus_int_bot} the root locus of the system with an PI-controller with gain $T_i = \frac{1}{0.035}$ and $K_p= 1$ is shown.

\begin{figure}[H]
\centering
\def\svgwidth{\textwidth}
\import{\main/afsnit/system_design/Controller/img/PI/}{rlocus_int_bot.pdf_tex}
\caption{Root locus of openloop plant with PI-controller}
\label{fig:rlocus_int_bot}
\end{figure}

A $K_p = 9$ will draw one of the poles from $0$ to the zero, and place a pole in $-5$, which will dominate the systems behavior. 
This is needed for the system to comply with the requirements, as explained in section \ref{sec:top_frame_tuning}.

\begin{figure}[H]
\centering
\def\svgwidth{\textwidth}
\import{\main/afsnit/system_design/Controller/img/PI/}{step_0_first_bot.pdf_tex}
\caption{Step response and bode plot of PI-controller on system}
\label{fig:step_0_first_bot}
\end{figure}


\begin{table}[h]
	 \centering
	 \begin{tabular}{ll}
		 \textbf{Overshoot} & 0.6323\%\\
		 \textbf{Risetime}  & 0.4143s\\
		  \textbf{Settling time}& 0.7133s
	 \end{tabular}
	 \caption{Performance characteristics and stability margins of PI-controlled system in figure \ref{fig:step_0_first_bot}.}
	 \label{tab:performance_0_5}
\end{table}


Figure \ref{fig:step_0_first_bot} then shows the step response of that PI-controller.
As in section \ref{sec:top_frame_tuning}, the response is not as expected, because the system is not a first order system. 
Therefore the P-gain is tuned to better meet the performance requirements. $K_p = 3.7$ returns the step response in figure \ref{fig:step_0_second_bot}.

\begin{figure}[H]
\centering
\def\svgwidth{\textwidth}
\import{\main/afsnit/system_design/Controller/img/PI/}{step_0_second_bot.pdf_tex}
\caption{Step response of tuned PI-controller on system}
\label{fig:step_0_second_bot}
\end{figure}


\begin{table}[h]
	 \centering
	 \begin{tabular}{ll}
		 \textbf{Overshoot} & 1.5803\%\\
		 \textbf{Risetime}  & 1.0226s\\
		 \textbf{Settling Time} & 1.6587s\\
	         \textbf{Gain Margin} & infinite\\
		 \textbf{Phase Margin} & 87.4\\
	 \end{tabular}
	 \caption{Performance characteristics of PI-controlled system in figure \ref{fig:step_0_second_bot}.}
	 \label{tab:performance_0_6}
\end{table}

The state space model of the bottom frame, when the top frame is at $90^\circ$, is given in equation \eqref{eq:bottom_state_eq_2}.

\begin{equation}
      \label{eq:bottom_state_eq_2}
      \begin{split}
      \frac{d}{dt}
    \begin{bmatrix}
        \theta_t \\
        \dot \theta_t \\
        i_t
    \end{bmatrix}
    =&
    \begin{bmatrix}0 & 1.0 & 0\\ 0 & -18.6 & 58.3\\ 0 & -9099.0 & -1.24\cdot10^4\end{bmatrix}
    \begin{bmatrix}
        \theta_t \\
        \dot \theta_t \\
        i_t \\
    \end{bmatrix}
    +
    \begin{bmatrix}
      0\\ 0\\ 2600
    \end{bmatrix}
    V_t
\\
      y =&
    \begin{bmatrix}
        1 & 0 & 0
    \end{bmatrix}
    \begin{bmatrix}
        \theta_t \\
        \dot \theta_t\\
        i_t\\
    \end{bmatrix}
    \end{split}
\end{equation}



In figure \ref{fig:comp_bot} a graph of both step responses are shown. It shows that there are no difference if the top frame is at an angle of $0^\circ$ or at $90^\circ$, when the two plants with the same controller is tested. Based on this, it is concluded that the changing moment of inertia does not have a relevant effect on the step response of the system.


\begin{figure}[H]
\centering
\def\svgwidth{\textwidth}
\import{\main/afsnit/system_design/Controller/img/PI/}{comp_bot.pdf_tex}
\caption{Comparison of the step responses of the two tuned PI-controllers with top frame at $0^\circ$ and $90^\circ$}
\label{fig:comp_bot}
\end{figure}



\end{document}
