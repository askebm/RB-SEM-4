\documentclass[../../../Main]{subfiles}
\begin{document}

In this section the controllers for the Top and Bottom frames of the pan-tilt system are designed and tuned, so they conform to the system requirements put forth in the problem statement.

\subsubsection{Tuning of Top-Frame}
\label{sec:top_frame_tuning}
The top frame is unlinear, as can be seen in equation \ref{eq:new_mech_equation_with_corners}, which includes a sine term.

To design a controller, for the topframe, the system is linearized in four points. It is done because the actual system step response looks more like a sawtooth pattern, than an actual sine pattern. See graph \ref{fig:jour_step_bot}

In this way, it is possible to have linear behaviour for the system in general.
Because $-90^o$ and  $+90^o$ are the has the same behavior in this system, only 3 linearizations are made.

\subsubsection{Linearize in $0^o$}
\label{sec:linearize}
The system is linearized in $0^o$. Which gives the following A, B and C matrices. 


\begin{equation}
      \begin{split}
      \label{eq:zero_linearized}
      \frac{d}{dt}
    \begin{bmatrix}
        \theta_t \\
        \dot \theta_t \\
        i_t
    \end{bmatrix}
    =&
    \begin{bmatrix}0 & 1.0 & 0\\ -10.8 & -75.8 & 212.0\\ 0 & -260.0 & -1.24\cdot10^{4}\end{bmatrix}
    \begin{bmatrix}
        \theta_t \\
        \dot \theta_t \\
        i_t \\
    \end{bmatrix}
    +
    \begin{bmatrix}
        0 \\
        0 \\
	\frac{1}{3.85\cdot10^{-4}}\\
    \end{bmatrix}
    V_t
\\
      y =&
    \begin{bmatrix}
        1 & 0 & 0
    \end{bmatrix}
    \begin{bmatrix}
        \theta_t \\
        \dot \theta_t\\
        i_t\\
    \end{bmatrix}
    \end{split}
\end{equation}

\begin{figure}[H]
\centering
\def\svgwidth{\textwidth}
\import{\main/afsnit/system_design/Controller/img/PI/}{rlocus_pure.pdf_tex}
\caption{Root locus of openloop plant without regulator.}
\label{fig:rlocus_pure}
\end{figure}

Based on the root-locus, the system can be regulated and and be stable only using a p-gain. 
But this will result in a steady-state error. therefore a PI-controller is tuned. 
A PI-controller is given by transferfunction \ref{equ:transfPI}

\begin{align}
	Kp \frac{s+\frac{1}{Ti}}{s}
	\label{equ:transfPI}
\end{align}
From equation \ref{equ:transfPI} it is seen, that a zero is added to the system.
and is placed in $-\frac{1}{Ti} $
The controller must be tuned to comply with the system requirements set forth in the problem statement, with a settling time of maximum 1 sec and no overshoot. For this to apply, the controlled system must be overdamped with real poles. And these poles must be located further to the left than $\sigma$
\begin{align}
	\sigma &\ge \frac{4.6}{ts}\\
	ts &= 1 sec\\
	\sigma &\ge 4.6
\end{align}

With a $T_i = \frac{1}{0.035}$ the root locus in figure \ref{fig:rlocus_int} is generated.

\begin{figure}[H]
\centering
\def\svgwidth{\textwidth}
\import{\main/afsnit/system_design/Controller/img/PI/}{rlocus_int.pdf_tex}
\caption{Root locus of openloop plant with integral regulator.}
\label{fig:rlocus_int}
\end{figure}

With a gain $Kp = 3.1$ a dominant pole is placed at $-5$. And the pole in 0, would approach the zero. Which should make this system mimic a first order system. The system controlled by a controller with gains $T_i = \frac{1}{0.035}$ and $Kp = 3.1$ returns a system with the stepresponse in figure \ref{fig:step_0_first}

\begin{figure}[H]
\centering
\def\svgwidth{\textwidth}
\import{\main/afsnit/system_design/Controller/img/PI/}{step_0_first.pdf_tex}
\caption{Step responce of system with the tuned PI-regulator. The dotted vertical line is 1 second.}
\label{fig:step_0_first}
\end{figure}

In figure \ref{fig:step_0_first} it can be seen that the settling time is faster than the expected 1 second. Therefore the $Kp$ and $T_i$ are tuned to better match the requirements. $Kp = 1.6$ and $T_i=0.2$ is found to give the closest to expectet response. The stepresponse can be seen in figure \ref{fig:step_0_second}.

\begin{figure}[H]
\centering
\def\svgwidth{\textwidth}
\import{\main/afsnit/system_design/Controller/img/PI/}{step_0_second.pdf_tex}
\caption{Step responce of system with the tuned PI-regulator. The vertical dotted line is 1 second.}
\label{fig:step_0_second}
\end{figure}

The regulated system has a slight overshoot, but has a risetime of 1 second as per the system requirements. The overshoot is the lowest possible, if the reference is to be reached. Furthermore the system takes 100 seconds to reach its final value.

\subsubsection{Linearize in $90^o$ and $180^o$}
In this section the linearized system is tuned in the same manor as is in section \ref{sec:linearize}. But not all steps are shown, as the approach is the same. Here the tuned values and the final stepresponses will be shown and commented.

Linearization in $90^o$ gives A, B and C matrices shown in equation \ref{eq:90_linearized}. 

\begin{equation}
      \label{eq:90_linearized}
      \begin{split}
      \frac{d}{dt}
    \begin{bmatrix}
        \theta_t \\
        \dot \theta_t \\
        i_t
    \end{bmatrix}
    &=
    \begin{bmatrix}0 & 1.0 & 0\\ 0  & -75.8 & 212.0\\ 0 & -260.0 & -1.24cdot10^4 \end{bmatrix}
    \begin{bmatrix}
        \theta_t \\
        \dot \theta_t \\
        i_t \\
    \end{bmatrix}
    +
    \begin{bmatrix}
     0\\ -1.57\\ 2600.0
    \end{bmatrix}
    V_t \\
		y &=
    \begin{bmatrix}
        1 & 0 & 0
    \end{bmatrix}
    \begin{bmatrix}
        \theta_t \\
        \dot \theta_t\\
        i_t
    \end{bmatrix}
    \end{split}
\end{equation}

Figure \ref{fig:step_0_second_90} shows the tuned stepresponse of the closed loop control sytem and plant.
It is tuned to $Kp = 1.39$ and $Ti = \frac{1}{0.006}$
\begin{figure}[H]
\centering
\def\svgwidth{\textwidth}
\import{\main/afsnit/system_design/Controller/img/PI/}{step_0_second_90.pdf_tex}
\caption{Step responce of system with the tuned PI-regulator.}
\label{fig:step_0_second_90}
\end{figure}


Linearization in $180^o$ gives A, B and C matrices shown in equation \ref{eq:180_linearized}. 

\begin{equation}
      \label{eq:180_linearized}
      \begin{split}
      \frac{d}{dt}
    \begin{bmatrix}
        \theta_t \\
        \dot \theta_t \\
        i_t
    \end{bmatrix}
    =&
    \begin{bmatrix}0 & 1.0 & 0\\ 10.8 & -75.8 & 212.0\\ 0 & -260.0 & -1.24\cdot10^4 \end{bmatrix}
    \begin{bmatrix}
        \theta_t \\
        \dot \theta_t \\
        i_t \\
    \end{bmatrix}
    +
    \begin{bmatrix}
    0\\ -1.0\\ 2600.0
    \end{bmatrix}
    V_t
\\
      y =&
    \begin{bmatrix}
        1 & 0 & 0
    \end{bmatrix}
    \begin{bmatrix}
        \theta_t \\
        \dot \theta_t\\
        i_t\\
    \end{bmatrix}
    \end{split}
\end{equation}


Figure \ref{fig:step_0_second_180} shows the tuned stepresponse of the closed loop control sytem and plant.
It is tuned to $Kp = 1.3$ and $Ti = \frac{1}{0.2}$
\begin{figure}[H]
\centering
\def\svgwidth{\textwidth}
\import{\main/afsnit/system_design/Controller/img/PI/}{step_0_second_180.pdf_tex}
\caption{Step responce of system with the tuned PI-regulator.}
\label{fig:step_0_second_180}
\end{figure}

The system responses are not exactly alike, and especially the system when linearized in $180^o$ does not look like the other stepresponses. In figure \ref{fig:comp}, all three responses are shown in the same plot, for comparison.

\begin{figure}[H]
\centering
\def\svgwidth{\textwidth}
\import{\main/afsnit/system_design/Controller/img/PI/}{comp.pdf_tex}
\caption{Step responce of all three PI-controlled systems}
\label{fig:comp}
\end{figure}

\subsubsection{tuning of bottom frame}

While the top frame of the pan-tilt system is unlinear, the bottom frame is not. Therefore it is not neccessary to linearize it. But as top frame can move, the moment of inertia, of the bottom- and topframe is a function of the top freames angle in relation to the bottomframe, as described in section\ref{}
\todo{reference here.}
In the following the regulator is tuned for ven the topframe is at $90^o$ and at $0^o$.

The statespace model of the bottomframe, when the topframe is at $0^o$  is found in equation \ref{:eq:bottom_state_eq}

\begin{equation}
      \label{eq:bottom_state_eq}
      \begin{split}
      \frac{d}{dt}
    \begin{bmatrix}
        \theta_t \\
        \dot \theta_t \\
        i_t
    \end{bmatrix}
    =&
    \begin{bmatrix}0 & 1.0 & 0\\ 0 & -75.8 & 238.0\\ 0 & -57.8 & -79.0\end{bmatrix}
    \begin{bmatrix}
        \theta_t \\
        \dot \theta_t \\
        i_t \\
    \end{bmatrix}
    +
    \begin{bmatrix}
      0\\ 0\\ 16.5
    \end{bmatrix}
    V_t
\\
      y =&
    \begin{bmatrix}
        1 & 0 & 0
    \end{bmatrix}
    \begin{bmatrix}
        \theta_t \\
        \dot \theta_t\\
        i_t\\
    \end{bmatrix}
    \end{split}
\end{equation}


\begin{figure}[H]
\centering
\def\svgwidth{\textwidth}
\import{\main/afsnit/system_design/Controller/img/PI/}{rlocus_pure_bot.pdf_tex}
\caption{Root locus of openloop plant without regulator.}
\label{fig:rlocus_pure_bot}
\end{figure}

In figure \ref{rlocus_pure_bot} a root locus of the openloop system is seen. As in section \ref{top_frame_tuning} the system can be controlled only by a p gain. But it will likely result in a steadystate error. Therefore the system is tuned with a PI-controller. In figure \ref{fig:rlocus_int_bot} the root locus of the system with an I-controller, is shown.

\begin{figure}[H]
\centering
\def\svgwidth{\textwidth}
\import{\main/afsnit/system_design/Controller/img/PI/}{rlocus_int_bot.pdf_tex}
\caption{Root locus of openloop plant without regulator.}
\label{fig:rlocus_int_bot}
\end{figure}

A gain of 9, will place a dominant pole in -5. which is needed for the system to comply with the requirements as explained in section \ref{sec:top_frame_tuning}.

\begin{figure}[H]
\centering
\def\svgwidth{\textwidth}
\import{\main/afsnit/system_design/Controller/img/PI/}{step_0_first_bot.pdf_tex}
\caption{Stepresponse of PI controller on system.}
\label{fig:step_0_first_bot}
\end{figure}

Figure \ref{fig:step_0_first_bot} then shows the stepresponse of that PI-controller.
As in section \ref{sec:top_frame_tuning}, the response is not quite as expected, as the system is not a first order system. Therefore the p gain is tuned to better meet requirements. $P = 3.7$ returns the stepresponse seen in figure \ref{fig:step_0_second_bot}.

\begin{figure}[H]
\centering
\def\svgwidth{\textwidth}
\import{\main/afsnit/system_design/Controller/img/PI/}{step_0_second_bot.pdf_tex}
\caption{Stepresponse of tuned PI controller on system.}
\label{fig:step_0_second_bot}
\end{figure}


\end{document}
