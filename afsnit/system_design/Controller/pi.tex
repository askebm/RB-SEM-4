\documentclass[../../../Main]{subfiles}
\begin{document}
\section{PI - Controller}
In this section the controllers for the Top and Bottom frames of the pan-tilt system are designed and tuned, so they conform to the system requirements put forth in the problem statement.

\subsection{Tuning of Top-Frame}

The top frame is unlinear, as can be seen in equation \ref{}, which includes a sine term.
\todo{ref}
To design a controller, for the topframe, the system is linearized in four points. It is done because the actual system step response looks more like a sawtooth pattern, than an actual sine pattern. See graph \ref{??}
\todo{insert reference}

In this way, it is possible to have linear behaviour for the system in general.
Because $-90^o and +90^o$ are the has the same behavior in this system, only 3 linearizations are made.

\subsubsection{Linearize in $0^o$}
\label{sec:linearize}
The system is linearized in $0^o$. Which gives the following A, B and C matrices. 


\begin{equation}
      \label{eq:zero_linearized}
      \begin{split}
      \frac{d}{dt}
    \begin{bmatrix}
        \theta_t \\
        \dot \theta_t \\
        i_t
    \end{bmatrix}
    =&
    \begin{bmatrix}0 & 1.0 & 0\\ -10.8 & -75.8 & 455.0\\ 0 & -7799.0 & -1.24\cdot10^{4}\end{bmatrix}
    \begin{bmatrix}
        \theta_t \\
        \dot \theta_t \\
        i_t \\
    \end{bmatrix}
    +
    \begin{bmatrix}
        0 \\
        0 \\
	\frac{1}{3.85\cdot10^{-4}}\\
    \end{bmatrix}
    V_t
\\
      y =&
    \begin{bmatrix}
        1 & 0 & 0
    \end{bmatrix}
    \begin{bmatrix}
        \theta_t \\
        \dot \theta_t\\
        i_t\\
    \end{bmatrix}
    \end{split}
\end{equation}

\begin{figure}[H]
\centering
\def\svgwidth{\textwidth}
\import{\main/afsnit/system_design/Controller/img/PI/}{rlocus_pure.pdf_tex}
\caption{Root locus of openloop plant without regulator.}
\label{fig:rlocus_pure}
\end{figure}

Based on the root-locus, the system can be regulated and and be stable only using a p-gain. 
But this will result in a steady-state error. therefore a PI-controller is tuned. 
A PI-controller is given by transferfunction \ref{equ:transfPI}

\begin{align}
	Kp \frac{s+\frac{1}{Ti}}{s}
	\label{equ:transfPI}
\end{align}
From equation \ref{equ:transfPI} it is seen, that a zero is added to the system.
and is placed in $-\frac{1}{Ti} $
The controller must be tuned to comply with the system requirements set forth in the problem statement, with a settling time of maximum 1 sec and no overshoot. For this to apply, the controlled system must be overdamped with real poles. And these poles must be located further to the left than $\sigma$
\begin{align}
	\sigma &\ge \frac{4.6}{ts}\\
	ts &= 1 sec\\
	\sigma &\ge 4.6
\end{align}

With a $T_i = \frac{1}{0.035}$ the root locus in figure \ref{fig:rlocus_int} is generated.

\begin{figure}[H]
\centering
\def\svgwidth{\textwidth}
\import{\main/afsnit/system_design/Controller/img/PI/}{rlocus_int.pdf_tex}
\caption{Root locus of openloop plant with integral regulator.}
\label{fig:rlocus_int}
\end{figure}

With a gain $Kp = 6.95$ a dominant pole is placed at $-5$. Which should make this system mimic a first order system. The system controlled by a controller with gains $T_i = \frac{1}{0.035}$ and $Kp = 6.95$ returns a system with the stepresponse in figure \ref{fig:step_0_first}

\begin{figure}[H]
\centering
\def\svgwidth{\textwidth}
\import{\main/afsnit/system_design/Controller/img/PI/}{step_0_first.pdf_tex}
\caption{Step responce of system with the tuned PI-regulator.}
\label{fig:step_0_first}
\end{figure}

In figure \ref{step_0_first} it can be seen that the settling time is faster than the expected 1 second. Therefore the $Kp$ is tuned to better match the requirements. $Kp = 2.82$ is found to give the closest to expectet response. The stepresponse can be seen in figure \ref{fig:step_0_second}.

\begin{figure}[H]
\centering
\def\svgwidth{\textwidth}
\import{\main/afsnit/system_design/Controller/img/PI/}{step_0_second.pdf_tex}
\caption{Step responce of system with the tuned PI-regulator.}
\label{fig:step_0_second}
\end{figure}

It is a bit slower than expected, but performes much closer to the requirements.


\subsubsection{Linearize in $90^o$ and $180^o$}
In this section the linearized system is tuned in the same manor as is in section \ref{sec:linearize}. But not all steps are shown, as the approach is the same. Here the tuned values and the final stepresponses will be shown and commented.

Linearization in $90^o$ gives A, B and C matrices shown in equation \ref{eq:90_linearized}. 

\begin{equation}
      \label{eq:90_linearized}
      \begin{split}
      \frac{d}{dt}
    \begin{bmatrix}
        \theta_t \\
        \dot \theta_t \\
        i_t
    \end{bmatrix}
    =&
    \begin{bmatrix}  0 & 1.0 & 0\\ 0 & -75.8 & 455.0\\ 0 & -7799.0 & -1.24\cdot10^4 \end{bmatrix}
    \begin{bmatrix}
        \theta_t \\
        \dot \theta_t \\
        i_t \\
    \end{bmatrix}
    +
    \begin{bmatrix}
     0\\ -1.57\\ 2600.0
    \end{bmatrix}
    V_t
\\
      y =&
    \begin{bmatrix}
        1 & 0 & 0
    \end{bmatrix}
    \begin{bmatrix}
        \theta_t \\
        \dot \theta_t\\
        i_t\\
    \end{bmatrix}
    \end{split}
\end{equation}

Figure \ref{fig:step_0_second_90} shows the tuned stepresponse of the closed loop control sytem and plant.
It is tuned to $Kp = 2.86$ and $Ti = \frac{1}{0.035}$
\begin{figure}[H]
\centering
\def\svgwidth{\textwidth}
\import{\main/afsnit/system_design/Controller/img/PI/}{step_0_second_90.pdf_tex}
\caption{Step responce of system with the tuned PI-regulator.}
\label{fig:step_0_second_90}
\end{figure}


Linearization in $180^o$ gives A, B and C matrices shown in equation \ref{eq:180_linearized}. 

\begin{equation}
      \label{eq:180_linearized}
      \begin{split}
      \frac{d}{dt}
    \begin{bmatrix}
        \theta_t \\
        \dot \theta_t \\
        i_t
    \end{bmatrix}
    =&
    \begin{bmatrix} 0 & 1.0 & 0\\ 10.8 & -75.8 & 455.0\\ 0 & -7799.0 & -1.24\cdot10^4 \end{bmatrix}
    \begin{bmatrix}
        \theta_t \\
        \dot \theta_t \\
        i_t \\
    \end{bmatrix}
    +
    \begin{bmatrix}
    0\\ -1.0\\ 2600.0
    \end{bmatrix}
    V_t
\\
      y =&
    \begin{bmatrix}
        1 & 0 & 0
    \end{bmatrix}
    \begin{bmatrix}
        \theta_t \\
        \dot \theta_t\\
        i_t\\
    \end{bmatrix}
    \end{split}
\end{equation}


Figure \ref{fig:step_0_second_180} shows the tuned stepresponse of the closed loop control sytem and plant.
It is tuned to $Kp = 2.8$ and $Ti = \frac{1}{0.035}$
\begin{figure}[H]
\centering
\def\svgwidth{\textwidth}
\import{\main/afsnit/system_design/Controller/img/PI/}{step_0_second_180.pdf_tex}
\caption{Step responce of system with the tuned PI-regulator.}
\label{fig:step_0_second_180}
\end{figure}

The system responses are not exactly alike, and especially the system when linearized in $180^o$ does not quite look like the other stepresponses. In figure \ref{}, all three responses are shown in the same plot for comparison.



\end{document}
