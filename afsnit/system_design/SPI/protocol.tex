\documentclass[../../../main]{subfiles}
\begin{document}
The SPI protocol is a shift register protocol with a possible register size of range from 8 bits to 16 bits \cite[sec.15]{tiva_data}.
The communication protocol is designed to keep the size of the data gram less or equal to the register size. This is to ensure a simple protocol between the Tiva and FPGA which is still able to utilize the full duplex capabilities of a SPI protocol. \\
The communication protocol uses the master-slave principle with the Tiva as the master and the FPGA as the slave.\\
When the Tiva initializes communication "Slave Enable" is driven low and one SPI clock cycle later,
data will be read on rising edge from the SPI clock, as seen in figure \ref{fig:spi_timing_diagram}.
\\
Each datagram is 16 bits in size and the datagram format sent by the Tiva
can be observed in table \ref{tab:package_format_tiva} and the package format send by the FPGA
can be observed in table \ref{tab:package_format_fpga}.
In the package format the bit number indicates the order of transmission starting with 0.

\begin{table}[h]
	\centering
	\begin{tabular}{ll}
		\textbf{Data}& \textbf{Data type}  \\
		\hline
		PWM& 9 bit signed \\
		Position& 12 bit signed \\
		Velocity& 12 bit signed \\
		Amps& 12 bit unsigned \\
		Home Index& 1 bit
	\end{tabular}
	\caption{Data types}
	\label{tab:spi_datatypes}
\end{table}
\begin{table}[h]
	\centering
	\begin{tabular}{|*{16}{p{0.50cm}|}}
		\hline
		0&1&2&3&4&5&6&7&8&9&10&11&12&13&14&15\\
		\hline
		\multicolumn{9}{|c|}{PWM  - 9bits} & M S&
		\multicolumn{2}{c|}{Resp. select}& R S&
		\multicolumn{3}{c|}{Reserved}
		\\
		\hline
	\end{tabular}
	\caption{Package format - Tiva}
	\label{tab:package_format_tiva}
\end{table}
\begin{table}[H]
	\centering
	\begin{tabular}{ll}
		MS & Motor Select\\
		\hline
		Response select &\\ &
		\begin{tabular}{ll}
			Position & 00\\
			Velocity & 01\\
			Acceleration & 10\\
			Amps & 11
		\end{tabular}
		\\\hline
		RS & Reset Position
		\\\hline
	\end{tabular}
	\caption{Package format extended}
	\label{tab:shorthand}
\end{table}

\begin{table}[H]
	\centering
	\caption{Package format - FPGA}
	\label{tab:package_format_fpga}
	\begin{tabular}{|*{16}{p{.3cm}|}}
		\hline
	 	0& 1& 2& 3& 4& 5& 6& 7& 8& 9& 10& 11& 12& 13& 14& 15\\
		\hline
		\multicolumn{2}{|p{.6cm}|}{Res} & H S 1 & H S 0 &
		\multicolumn{12}{c|}{Requested Data}\\
		\hline
	\end{tabular}
\end{table}

\begin{figure}[h]
	\center
\begin{tikztimingtable}[timing/font=\normalfont]
	{2 MHz Clock}&1.5L31{t}1L\\
	{Slave Enable}&1H16L1H\\
	{MOSI [MSG \#N]}&1L16{D{}}1L\\
	{MISO [MSG \#N-1]}&1U16{D{}}1U\\
\end{tikztimingtable}
\caption{SPI timing diagram}
\label{fig:spi_timing_diagram}
\end{figure}

\end{document}
