\documentclass[../../../main]{subfiles}
\begin{document}
 
To be able to set a desired angel for the system, it is necessary to have a reference point on which the system always can return to. 
The system is equipped with a hall sensor in both the top frame and the bottom frame.
These sensors are used to determine when the frame is at $0^\circ$. The position of the sensors and end stop can be seen on figure \ref{fig:homing_system}.
\\
For the system to be able to return to $0^\circ$ a homing processes needs to be designed.
The bottom frame cannot turn freely, this needs to be taken into consideration when designing the process.
\subsubsection{Methods} \todo{need a proper title at some point}
There is multiple ways to home the system. It can be done manually by rotating both the frames to the home position.
For this to work the FPGA has to be reset manually after the frames are in position.
This is not the best way of doing so, since it is not possible to home the system without turning it off and the dc-motors has to be turned manually.
An automated homing process is preferred, and two methods have been considered to achieve this. 
\begin{figure}[H]
  \centering
  \def\svgwidth{\textwidth}
  \includesvg[\main /afsnit/system_design/homing/img/]{system_homing}
  \caption{Placement of sensors and end stop.}
  \label{fig:homing_system}
\end{figure}
\paragraph{Method 1}%
\label{par:method_1}
One way of homing the system is to give the top frame a low PWM signal, so that it is rotating slowly. 
When the top frame triggers the hall sensor, it resets the encoder for the top frame. 
Then the reference for the controller is set to zero, so that the frame moves to $0^\circ$. 
\\
For the bottom frame it is necessary to be able to determine if the frame is hitting the end stop. 
If it is hitting the end stop, it needs to switch direction of the motor for the frame to be able to return to home position.
The detection of the end stop can be done by checking if the frame still is rotating, if not then switch direction of the motor.
Otherwise the homing of the bottom frame will be identical to the top frame.

\paragraph{Method 2}%
\label{par:method_2}
Another method of homing the system is to rotate the frames slowly, like in the previous method. 
Instead of resetting the encoders when the frames trigger the hall sensors, it will save the position in witch the sensors was triggered, and then change direction of the motor.
When the sensor is triggered again it will save the that position, and calculate the difference between the to potions. 
Then the reference for the controller will be the calculated potions, and when the frames are triggering the sensors for a give time the encoders will be reset.
\\
This approach will most likely be more precise, but also more complicated. 



\subsubsection{Homing summary}%
\label{ssub:homing_summary}
For this system the second method has been chosen. This is done because a precise home position is desired.
The system is also able to detect if the bottom frame is hitting the end stop. If this happens while homing the direction of the bottom frame is switched.
This makes the system able to start at the same position every time.
If for some reason the position should drift under use, the system is also able to reset the position on request.





\end{document}
