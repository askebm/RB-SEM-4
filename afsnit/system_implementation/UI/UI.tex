\documentclass[../../../main]{subfiles}
\begin{document}
To be able to set a reference for the system, and to see where the system is at any given time, a user interface needs to be implemented.
There are multiple ways in which a user interface can be implemented. 
Two will be discussed in this section, a command-line interface and a hardware interface.

\subsubsection{Command-line interface}%
\label{ssub:command-line_interface}
A command-line interface is practical for designing and tuning of the system, since it allows for data to be sent to a computer and analyzed.
It is not the most user friendly way to interact with a system, and therefor it should not stand on its own.
\\
Logging data is essential for tuning of the system, and therefore a command-line interface has been implemented. 
The interface commands can be seen in table XX. \todo{create table}
\\
Implementation of the command-line interface is done by utilizing the USB port on the Tiva to communicate with a computer via a UART interface.  

\subsubsection{Hardware interface}%
\label{ssub:hardware_interface}
A hardware interface is more user friendly, and it is possible to use without connecting a computer to the system. 
Since the system already is equipped whit both a matrix keypad and a small LCD screen, these have been used to create a hardware interface.
\\
Since the LCD screen only can display 16X2 characters only the most essential information is shown.


\begin{figure}[H]
  \centering
  \def\svgwidth{\textwidth}
  \includesvg[\main /afsnit/system_implementation/UI/img/]{UI_emp_board}
  \caption{User interface.}
  \label{fig:User_interface}
\end{figure}


\subsubsection{User interface summary}%
\label{ssub:user_interface_summary}



\begin{itemize}
    \item Why do we need a UI? 
    \item Descripe UART UI
        \begin{itemize}
            \item protcol table
        \end{itemize}
    \item Descripe LCD UI
    \item Conclude on UI 
\end{itemize}


\end{document}
