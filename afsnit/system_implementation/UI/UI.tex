\documentclass[../../../main]{subfiles}
\begin{document}
To be able to set a reference for the system, and to see where the system \todo{ positions are?} is at any given time, a user interface needs to be implemented.
There are multiple ways in which a user interface can be implemented.
Two will be discussed in this section, a command-line interface and a hardware interface.

\subsubsection{Command-line interface}%
\label{ssub:command-line_interface}
A command-line interface is practical for designing\todo{designing?} and tuning of the system, since it allows for data to be sent to a computer and analyzed.
\todo{however }It is not the most user friendly way to interact with a system , \todo{and should therefore not stand alone, but used as a debug-tool}and therefor it should not stand on its own and is more suited for debug operations.
\\
Logging data is essential for tuning of the system, and therefore a command-line interface has been implemented.
\todo{Since the desired system response is achieved through tuning of the controller gains, it is necessary to have acces to the system response data.
For this reason it is chosen to implemnt command-line interface. }
\\
Implementation of the command-line interface is done \todo{is realised by utilizing ...} by utilizing the USB port on the Tiva \todo{. This is then used to communicate with a computer via a UART interface.} to communicate with a computer via a UART interface.

\subsubsection{Hardware interface}%
\label{ssub:hardware_interface}
\todo{hvad er argumentationen for dette?}
A hardware interface is more user friendly, \todo{furthermore it doesnt require a computer connected to the system }and it is possible to use without connecting a computer to the system.
Since the system already is equipped with a matrix keypad and a small LCD screen, \todo{these will be used to create the hardware interface }these have been used to create a hardware interface.
\\
\todo{However the LCD is only capable of displaying 32 characters ( or 16x2), thus only the position and refernce is displayed}
Since the LCD screen only can display 16X2 characters only the most essential information, the current position and reference is shown.


\begin{figure}[H]
  \centering
  \def\svgwidth{\textwidth}
  \includesvg[\main /afsnit/system_implementation/UI/img/]{UI_emp_board}
  \caption{User interface.}
  \label{fig:User_interface}
\end{figure}


\subsubsection{User interface summary}%
\label{ssub:user_interface_summary}

These two systems provide a way to interact with the system, either in order to extract data or change the systems reference point.
\todo{A command-line interface has been implemented in order to acces the system response data.
Furhtermore, a hardware interface has been implemted. This is to use the system when the tuning is done, and the debug utilities no longer is required.
These two interfaces provide the option to tune and use the system as desired.}
\end{document}
