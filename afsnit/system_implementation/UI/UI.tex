\documentclass[../../../main]{subfiles}
\begin{document}
To be able to set a reference position for the system, and to see where the system currently is, a user interface needs to be implemented.\\
There are multiple ways in which a user interface can be implemented.
Two of which will be discussed in this section, a command-line interface and a hardware interface.

\subsubsection{Command-line interface}%
\label{ssub:command-line_interface}
A command-line interface is practical for tuning the system, since it allows for data to be sent to a computer and analyzed.
However It is not suitable as a user friendly way to interact with the system on its own.
\\
Since logging data is essential for tuning of the system, a command-line interface has been implemented to achieve a test for the regulator gains.
The command-line interface is also used for debugging purpose.
\\
Implementation of the command-line interface is realized by utilizing the USB port on the Tiva to communicate with a computer via a UART interface.

\subsubsection{Hardware interface}%
\label{ssub:hardware_interface}
The hardware interface is considered more user friendly since it is usable without connecting a computer to the system.
Since the system already is equipped with a matrix keypad and a small LCD screen, these will be used to create the hardware interface.
\\
However the LCD is only capable of displaying $32$ characters distributed on $2$ rows of 16 characters, thus only the position and reference is displayed.
\\
The matrix keypad is used to change the reference for the system, by typing in the position in degrees first for the top frame followed by a pound sign. 
After setting a reference for the top frame it is possible to set a reference to the bottom frame with the same procedure. 

\begin{figure}[H]
  \centering
  \def\svgwidth{\textwidth}
  \includesvg[\main /afsnit/system_implementation/UI/img/]{UI_emp_board}
  \caption{User interface.}
  \label{fig:User_interface}
\end{figure}

\end{document}
