\documentclass[../../../main]{subfiles}


\begin{document}


\subsection{Softwaredesign of a PI controller}
In order to design and implement the aforementioned PI-controller in a fashion that complies with API from FreeRTOS.
It is chosen that a single interface is to be designed so that only one subprogram has to be written, thus it requires to be generic, if multiple multiple PI-controllers are to be implemented.
To meet the given requirements the following design criteria must be met:

\begin{itemize}

  \item Must comply with FreeRTOS's API
  \item Must be re-entrant
  \item Must be able to be initialized from main
  \item Must be generic
\end{itemize}


In order to write a routine that will act as the PI-controller, it must be wrapped into a never ending loop, in which the desired calculations will be done, this is done to meet the first criteria.
This is achieved in listing \ref{list:funtion_body}.

\lstinputlisting[language=C, firstline=3, lastline=16,label={list:funtion_body} ]{code_snippets.h}

This will be the function which the task created by FreeRTOS will have as an argument.
It is noticed that all of the above criteria are met, if the functions calculate\_error\_PI() and calculate\_P() are assumed to be re-entrant.
For it to be generic the struct PI, is passed from the function, task\_PI(), this also ensures that it can be initialized from main.
\\

The struct will contain all the necessary gains and data-samples, which a PI-controller will need in order to operate.
The struct members can be seen in \ref{list:struct_declar}.
\lstinputlisting[language=C, firstline=18, lastline=31,label={list:struct_declar}]{code_snippets.h}


Since the design requirements are met, the next step is to implement the funtionallity of a PI-controller, which has the following requirements:
\begin{itemize}
    \item Must be implemented as a difference equation
    \item Must include anti-windup
\end{itemize}

The reasoning for it to be implemented as a difference equation, is due to it being the easiest and most efficient way to compute the controller-signal.

\subsection{Deriving the PI difference equation}

In order to obtain the difference equation for the PI-controller, the standard PI equation in the s-domain is observed:

\begin{equation}
  u(s) = \Bigg(k_p + \frac{k_i}{s} \Bigg) \cdot e(s)
\end{equation}


To get the PI equation into the time-discrete z-domain, it is chosen to use the Tustin's Method.

This is done by substiting $s$ with the following:
$$
s = \frac{2}{T}\cdot \Bigg( \frac{z-1}{z+1}\Bigg)
$$

Equation ref??  now becomes
$$
  u(z) = \Bigg(k_p + k_i\cdot \frac{T}{2} \frac{z+1}{z-1} \Bigg) \cdot e(z)
$$















\end{document}
