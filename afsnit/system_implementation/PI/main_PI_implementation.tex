\documentclass[../../../main]{subfiles}

\begin{document}
\lstset{basicstyle=\tt\scriptsize}

\lstset{language=C,
	tabsize=4,
	numbers=left,
	numberstyle=\tiny,
	keywordstyle=\bfseries\color{green!40!black},
    commentstyle = \color{olive},
	stepnumber=1,
	numbersep=5pt}
\lstset{morekeywords={int16\_t, uint8\_t, PI}}

\subsection{PI}

\subsubsection{Software design of a PI controller}
In order to implement the aforementioned PI-controller in a fashion that complies with the API provided by FreeRTOS, it has been chosen to design a single interface.
This is to avoid writing multiple sub-programs with the same functionality, and thus, it is required to be generic if multiple PI-controllers are to be implemented.
\\

To meet the requirements, the following design criteria must be fulfilled.

\begin{itemize}
  \item Must comply with FreeRTOS's API
  \item Must be re-entrant
  \item Must be able to be initialized from main
  \item Must be generic
\end{itemize}

In order to write a routine that will act as the PI-controller, it must be wrapped into a never ending loop.
In this loop the desired calculations will be done. This is done to meet the first criteria and is achieved in listing \ref{list:funtion_body}.

\lstinputlisting[ firstline=3, lastline=16,caption = {Function body} ,label={list:funtion_body} ]{\main/afsnit/system_implementation/PI/code_snippets.h}



tsk\_PI() will be the argument to the function by FreeRTOS, that creates a task.
It is noticed that if the functions, calculate\_error\_PI() and calculate\_P(), are assumed to be re-entrant and the third and fourth criteria are achieved by passing the struct PI from the function task\_PI(), all of the above criteria are achieved.
%For it to be generic the struct PI, is passed from the function, task\_PI(), this also ensures that it can be initialized from main.
\\

The struct will contain all the necessary gains and data-samples, which a PI-controller needs in order to operate.
The struct members can be seen in listing \ref{list:struct_def}.
\lstinputlisting[ firstline=20, lastline=33,caption = {Struct definition} ,label={list:struct_def}]{\main/afsnit/system_implementation/PI/code_snippets.h}

Since the design requirements are met, the next step is to implement the functionality of a PI-controller with the following requirements:
\begin{itemize}
    \item Must be implemented as a difference equation
    \item Must include anti-windup
\end{itemize}
The reasoning for it to be implemented as a difference equation, is due to it being an exact relationship between current and past samples.
%The reasoning for it to be implemented as a difference equation, is due to it being the easiest and most efficient way to compute the controller-signal.

\subsubsection{Deriving the PI difference equation}

In order to obtain the difference equation for the PI-controller, the standard PI equation in the s-domain is observed:

\begin{equation}
  u(s) = \Bigg(k_p + \frac{k_i}{s} \Bigg) \cdot e(s) \label{eq:eq_1}
\end{equation}


To get the PI equation into the time-discrete z-domain, it is chosen to use Tustin's Method.

This is done by substituting $s$ with the following

$$
s = \frac{2}{T}\cdot \Bigg( \frac{z-1}{z+1}\Bigg)
$$

Equation \eqref{eq:eq_1} now becomes
$$
  u(z) = \Bigg(k_p + k_i\cdot \frac{T}{2} \frac{z+1}{z-1} \Bigg) \cdot e(z)
$$
Now the expression on the right side is refactored

$$
  u(z) = \Bigg( \frac{k_p \cdot 2(z-1) + k_i \cdot T \cdot (z+1)}{2\cdot (z-1)}\Bigg) \cdot e(z)
$$
In order to have the discrete equation in terms of previous samples, both sides of the fraction are divided by $z^{-1}$

$$
  u(z) = \Bigg( \frac{k_p \cdot(1-z^{-1}) + k_i \cdot \frac{T}{2} \cdot (1+z^{-1}) }{(1-z^{-1})}\Bigg) \cdot e(z)
$$

It is wished to get an expression in terms of u(z) and e(z) with no non-constant-fractions, therefore both sides are multiplied by $(1 - z^{-1})$
$$
  u(z)\cdot(1 - z^{-1}) = \Bigg( k_p \cdot(z-1) + k_i \cdot \frac{T}{2} \cdot (z+1) \Bigg) \cdot e(z)
$$

The inverse z-transformation is now performed on both sides

$$
  u[n]-u[n-1] =  k_p \cdot (e[n] - e[n-1])  + k_i \cdot \frac{T}{2}(e[n] + e[n-1])
$$
The final difference equation becomes

$$
  u[n] = u[n-1] + k_p \cdot (e[n] - e[n-1])  + k_i \cdot \frac{T}{2}(e[n] + e[n-1])
$$



\subsubsection{Implementation of the difference equation}
With the software design chosen and the difference equation derived, the next step is to implement it.
This is done in the following code snippet,seen in listing \ref{list:PI_controller}, which includes anti-windup whilst having the PI-controller expressed as a difference equation.

\lstset{morekeywords={int16\_t, uint8\_t, PI,x\_n,y\_n}}
\lstinputlisting[ firstline=37, lastline=67,caption = {PI controller, with \text{x\_n =  e[n] and y\_n = u[n]} } ,label={list:PI_controller}]{\main/afsnit/system_implementation/PI/code_snippets.h}






\end{document}
