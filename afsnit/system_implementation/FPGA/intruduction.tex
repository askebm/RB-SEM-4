\documentclass[../../../main]{subfiles}
\begin{document}
The purpose of this section is to discuss some of the challenges and benefits that are related to the use of a FPGA in this project.
The FPGA connects the microcontroller to all the components on the pan-tilt system in order to control the system. It collects data from encoders and homing sensors as well as creating and updating the PWM signal to both of the H-Bridges.
\\
One of the main differences between a FPGA and a microcontroller is that on a FPGA everything is implemented in hardware. This means that all of the processes are running in parallel.
When \todo{This needs to be taken into consideration when implementing the various components and global registers so that a register can not be updated while other processes are using them} implementing the different components this needs to be taken into consideration, so that registers issent updatet while other processes are reading them.
\\
The FPGA used in this project is a Artix-7 mounted on a BASYS3 kit \cite{xilinx2019fpga}.
\subsubsection{System overview}%
\label{sub:system_overview}

\begin{figure}[H]
  \centering
  \def\svgwidth{\textwidth}
  \includesvg[\main /afsnit/system_implementation/FPGA/images/]{components_connection}
  \caption{FPGA System overview.}
  \label{fig:FPGA_system_overview}
\end{figure}
The system consists of different modules mplemented in VHDL.
The data paths between the modules can be seen in the system overview, see figure \ref{fig:FPGA_system_overview}.
\\
To \todo{mit bud: Multiple registers, of different sizes depending on the useage, have been implemented to store the data that is to be exhanged between the FPGA and the Tiva, see table \ref{table:FPGA_registers}.} store the data that needs to be exchanged between the FPGA and the Tiva microcontroller, multiple registers is implemented on the FPGA with the necessary size. See table \ref{table:FPGA_registers}.
\\
\begin{table}[H]
\centering
\begin{tabular}{|c|c|c|}
\hline
\textbf{$12$ Bit} & \textbf{$9$ Bit} & \textbf{$1$ Bit} \\ \hline
Position\_T     & PWM\_T         & Home\_T        \\ \hline
Position\_B     & PWM\_B         & Home\_B        \\ \hline
Velocity\_T     & -              & Reset\_T       \\ \hline
Velocity\_B     & -              & Reset\_B       \\ \hline
\end{tabular}
\caption{FPGA Registers}
\label{table:FPGA_registers}
\end{table}

\subsubsection{PWM}
The PWM module creates a PWM signal for both motors. It takes a 9 bits vector as input for each motor, and output the signals necessary for the H-Bridges to work.
It \todo{igen, lille ændring hvis du kan lide den: It takes the 100MHz system clock from the FPGA as input which it uses to create a downscaled clock frequency for the desired PWM frequency.} takes the 100Mhz system clock from the FPGA as input and, as the module contains a clock divider it creates the necessary clock frequency for the desired PWM frequency.
MSB in the 9 bit input vector determines which way the motor should turn.
The module is designed such that it is possible to define the system clock frequency, the desired PWM frequency and the correct clock divider is calculated when the code is synthesized.
This is done so that it is easy to change the PWM frequency for testing.
\subsubsection{Protocol}
The protocol module uses a SPI module created by \todo{ref.} for communication with the Tiva.
The role of the protocol module is to decode the data received over SPI, and prepare data for transmission, in regards to the determined protocol in section \ref{sub:spi}.
\\
Since all the different processes, like receiving data over SPI and updating data registers, run in parallel, it is important to make sure that the protocol module cannot update the data registers while the SPI module is receiving data form the Tiva.
This is done by placing a latch on the output from the protocol module, which is triggered by a busy signal from the SPI module.

\subsubsection{Quadrature decoder}
\label{subsubsec:Qdecoder_implement}
Since there are two motors on the system and therefore two encoders, two instances of the quadrature decoder module has been implemented.
Each of these modules has a 12 bit position vector as thier output and takes the A and B encoder signals directly from the encoder. The modules uses the 100Mhz system clock directly and takes a reset which is active low as input.
The reset for the encoders are inverted before they are handed to the modules, so when a high reset is sent by the Tiva, a low reset signal is given to the decoder.


\subsubsection{Speed measurement}
Thee two modules are used to calculate the speed of which the system is rotating. For this calculation it uses the 12 bit output vector from the quadrature decoder module.
There is two speed measurement modules implemented, one for each motor.
This is done to make it possible to regulate both the position and speed for future developmenting of the system.

\subsubsection{FPGA Summery}
All of the above mentioned modules have been developed and implemented on the Artix7 FPGA.
The FPGA is able to communicate with the Tiva using SPI and the designed protocol.
\\
The big advantage of using a FPGA for controlling the motors and reading the encoders is that all of the modules are implemented in hardware.
This means that all of the processes run in parallel and therefore it is not necessary to take timing of processes into consideration.


\end{document}
