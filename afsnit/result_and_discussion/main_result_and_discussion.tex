\documentclass[../../main]{subfiles}
\begin{document}

\section{Modelling}

\subsection{General model}
\label{sec:General_model}

In this chapter an approximated physical model of the system will be found.\\

By analysis of the forces acting on the motors when voltage is added, the following differential equations are found.\\

\begin{equation}
  \label{equ:model_mech_equ}
  J\cdot \ddot \theta + b\cdot \dot \theta = K_t\cdot i
\end{equation}

\begin{equation}
  \label{equ:model_ele_equ}
  L\cdot \frac{di}{dt} + R\cdot i = V - K_e\cdot \dot \theta
\end{equation}

In (\ref{equ:model_mech_equ}) $J$ is the moment of inertia about the axis of rotation, $b$ is the motor viscous friction constant, $K_t$ is the motor toque constant, $\dot \theta$ is the first time derivative of the position, and $\ddot \theta$ is the second time derivative. The first time derivative is this the velocity and the second is the acceleration.\\
In (\ref{equ:model_ele_equ}) $L$ is the electric inductance of the coil in the motor, $\frac{di}{dt}$ is the first time derivative of the current, $R$ is the ohmic resistance of the motor, $V$ is the voltage applied, $K_e$ is the electromotive force constant and as mentioned above $\dot \theta$ is the velocity.\\
These can now be written in state-space form as shown in eq. (\ref{equ:model_ss_1}) and (\ref{equ:model_ss_2}).

\begin{equation}
      \label{equ:model_ss_1}
      \frac{d}{dt}
    \begin{bmatrix}
        \dot \theta \\
        i
    \end{bmatrix}
    =
    \begin{bmatrix}
        -\frac{b}{J} & \frac{K}{J}\\
        -\frac{K}{L} & -\frac{R}{L}\\
    \end{bmatrix}
    \begin{bmatrix}
        \dot \theta \\
        i \\
    \end{bmatrix}
    +
    \begin{bmatrix}
        0 \\
        \frac{1}{L} \\
    \end{bmatrix}
    V
\end{equation}

\begin{equation}
      \label{equ:model_ss_2}
      y =
    \begin{bmatrix}
        1 & 0
    \end{bmatrix}
    \begin{bmatrix}
        \dot \theta\\
        i\\
    \end{bmatrix}
\end{equation}

\subsection{Application specified model}

To apply the models found in the section \textit{General model} on page \pageref{sec:General_model}, the different contants must be determined. Since the pan-tilt system contains two motors, one for the top-frame, and one for the base-frame. The constants concerning these will from this point on be denoted with the following subscripts in the same order as mentioned above: $K_t$ and $K_b$.\\
Firstly the ohmic resistance of the two motors can be seen determined in jounal \textit{Determine ohmic resistance of mortors} as $R_t = 4.74 \Omega$ and $R_b = 4.70 \Omega$. The inductance of the motors now also has to be determined. This can be seen done in the journal \textit{Determine electrical inductance of motors}. Here they can be seen as $L_t = L_b = 3.85\cdot 10^{-4}$.


























\end{document}
