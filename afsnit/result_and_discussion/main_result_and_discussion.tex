\documentclass[../../main]{subfiles}
\begin{document}

In this section, the reasons for the differnece in the found theoretical model and the observed behavior of the system will be discussed together with suggstions to improve the project.
% In the following section titles are gonna appear, BE AWARE that these only serve the purpose of structuring the section and will be deleted after the writing is done!
\subsection{Differnce in practical vs. theoretical results}
Fistly the above mentioned difference must be quantified. This can be seen on figure REF\\
\todo{Figure of difference in theoretical and practical data}\\
To diagnose the potential causes for this difference, its worth considering the approximations made in process of simulating the pratical system.\\
As seen in section \ref{sec:Top_frame_inertia} the moment of inertia is estimated through the following approximations: The cornors on the top frame are considered added mass to $d$ while their center of mass really are closer to the axis of rotation, the bolts keeping the frames together aren't accounted for together with the asymmetric added mass of the motors mounted on the frames. Since moment of inertia is an expression of mass distribution about a given axis of rotation, we should expect that these approximations would indicate a potential error with same sign as the one we observe. As we would expect a greater moment of inertia would result in a bigger overshoot. This can be seen on figure REF, which from the above mentioned approximations confirm since the motor weights together with the bolts in the frame would add to the moment of inertia. This is to some extend counter weighed by the corners which would contribute to the moment of inertia to a smaller overshoot. It is here worth mentioning that the displacement of the frame corners would contribute to the lowering of the overshoot. While this might be present, its influence on the resulting behavior seems to be small based on the systems behavior in practice.\\
One onther constant which might differ from the real one is the ohmic resistance in the motors. From the datasheet of the used multimeter BIBREF its seen that the percentage error generally is $0.0075\%$ which results in the correct resistance for the top frame motor being in the interval $[4.7346\Omega;4.7534\Omega]$ and for the bottom frame its $[4.7016\Omega;4.7024\Omega]$. Since these intervals are as small as shown, they would not contribute in a significant way to the overshoot, and are therefore viewed as negliable.\\
\subsection{Improvements}
To improve the control of the system, implementation of a D-term followed by a lowpass filter would be an option, since the D-term in a PID controller reduces the amouont of overshoot.







\end{document}
