\documentclass[../../main]{subfiles}
\begin{document}


The overarching aim of the project is to create a platform for a stage spotlight with the behavior expected of such a system in regards to movement characteristics.

As the mechanical platform is provided for the project, the main challenges arise from the theoretical system, control design and software engineering required to materialize the platform.

These challenges include:

\begin{itemize}
    \item What are the desired performance requirements for a spotlight?
    \item In what ways can a system of this type be controlled?
    \item How are control engineering theory implemented on a microcontroller?
    \item How can a system allowing hardware control of the given platform be designed?
\end{itemize}

In order to provide a satisfactory answer to the above questions, the design of the FPGA modules, microcontroller software and control engineering theory will be covered in the following report.

\subsection{System requirements}

The theoretical usecase of the system is as mentioned a stage spotlight. The system requirements have be rather hard.
Overshoot is undesirable as a spotlight is placed rather far away from the stage. Even a $1\%$ overshoot would result in a sizeable overshoot of the intented target which could potentially be disruptive to the stage show.
Furthermore the system needs a rise time as fast as possible without overworking the motors and the gearing.

A rise time of $1$ second fits both criterias, as this should not need a unrealitic amount of current for the system to react nor should it feel too slow and cause a noticable lag to the stage light.
The settling time is set to be $1.2$ seconds as this gives the system time to slow down when it is nearing the target and settling in to $1\%$ of the reference.
\\
The requirements can be observed in the list below:

\begin{itemize}
  \item No overshoot.
  \item Maximum 1 second risetime.
  \item 1 \% settling time of 1.2 seconds.
\end{itemize}

\end{document}
