\documentclass[../../main]{subfiles}
\begin{document}


The overarching aim of the project is to create a platform for a stage spotlight with the behavior expected of such a system in regards to movement characteristics.

As the mechanical platform is provided the project can be resolved using the fields of study connected to this project, which are; control engineering, digital system design and embedded programming.
The problem statement is therefore divided by field of study.

%for the project, the main challenges arise from the theoretical system, control design and software engineering required to materialize the platform.
%These challenges include:

%\begin{itemize}
%    \item What are the desired performance requirements for a spotlight?
%    \item In what ways can a system of this type be controlled?
%    \item How are control engineering theory implemented on a microcontroller?
%    \item How can a system allowing hardware control of the given platform be designed?
%\end{itemize}

\paragraph{Control Engineering}%
\label{par:control_engineering}
Model the system such that a controller can be designed and implemented from said model.
Design a controller such that the performance specifications are met in a satisfactory manner.

\paragraph{Digital System Design}%
\label{par:digital_system_design}
Make the FPGA able to communicate by way of SPI and implement a protocol.
Measure velocity and position reliably, such that these can be used to monitor and regulate the
system.
Control motors using PWM.

\paragraph{Emebedded Programming}%
\label{par:emebedded_programming}
Implement a real time operating system such that it will be able to handle user input while simultaneously controlling the given system consistently.
\\

To properly define the project a set of performance specification regarding the controller design is necessary.
These specifications consists of rise time, which is the time it takes to reach a given set point with a threshold.
Settling time, which is the time it takes to stay within 1\% of the set point indefinitely.
Lastly there is overshoot the percentage amount with a system initially surpasses its set point.\\

%In order to provide a satisfactory answer to the above questions, the design of the FPGA modules, microcontroller software and control engineering theory will be covered in the following report.



%\subsection{System requirements}


As the use case of the system is as a stage spotlight, the performance specifications  need to be rather hard as overshoot and a slow rise time could render the system useless.
Overshoot is undesirable as a spotlight is placed rather far away from the stage. Even a $1\%$ overshoot would result in a sizable overshoot of the intended target which could potentially be disruptive to the stage show.
Furthermore the system needs a rise time as fast as possible without overworking the motors and gearing.

A rise time of $1$ second fits both criteria, as this should not need a unrealistic amount of current for the system to react nor should it react so slow it causes a noticeable lag to the stage light.
The settling time is set to be $1.2$ seconds as this gives the system time to slow down when it is approaching the target and settling in to $1\%$ of the reference.
To limit the scope of these specifications they will only apply for movement deltas up to 90$^o$
\\
The performance specifications for a 90$^o$ movement are listed below:

\begin{itemize}
  \item No overshoot.
  \item Maximum $1s$ rise time.
  \item $1\%$ settling time of $1.2s$.
\end{itemize}
\end{document}
