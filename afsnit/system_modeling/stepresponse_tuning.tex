\documentclass[../../main]{subfiles}
\begin{document}

To tune the theoretical model the electromotiveforce, the motortorque and friction, is adjusted for both the top frame and the bottom frame, to fit the openloop stepresponse of the physical system.

Further more the total weight of the corners is increased by 200\si{g}. The bottom frame also have what seems to be a sizeable nonlinear component, which is evaluated to be a torque of $1.4$\si{N m} sine dependency on the bottom frames position.

The adjusted motorconstants can be viewed in table \ref{tab:fitted_motor_constants} and the resulting simulated stepresponse interlaced with the physical stepresponse can be seen in figure \ref{fig:openloop_step_phys}.

\begin{table}[h]
	\centering
	\begin{tabular}{ll}
		\multicolumn{2}{c}{ \textbf{Top frame} }\\
		\hline
		electromotiveforce& 3\si{ \frac{rad}{Vs}}\\
		motortorque & 2.8 \si{ \frac{N m}{A}}\\
		friction coefficient& 0.5\\
		m$_{corners}$& 0.351\si{g}  \\

		\multicolumn{2}{c}{ \textbf{Bottom frame} }\\
		\hline
		electromotiveforce& 3\si{ \frac{rad}{Vs}}\\
		motortorque & 4.4 \si{ \frac{N m}{A}}\\
		friction coefficient & 0.8\\
		non-linear torque & 1.4\si{N m}\\
	\end{tabular}
	\caption{Fitted motor constants.}
	\label{tab:fitted_motor_constants}
\end{table}

\begin{figure}[h]
	\centering
	\def\svgwidth{\textwidth}
	\import{\main/afsnit/system_modeling/img/}{step_response.pdf_tex}
	\caption{Openloop stepresponse of physical and simulated system.}
	\label{fig:openloop_step_phys}
\end{figure}

The resulting simulated model is somewhat close, but there are stille some inconsistencies which could be exlained by slack in the system.

\end{document}
