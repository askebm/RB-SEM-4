\documentclass[../../main]{subfiles}
\begin{document}
As the problem statement is divided into field of studies, the conclusion is written in the same manner.

\paragraph{Control engineering}

A mathematical model of the pan-tilt system has been made.
It is noticed that in this model, both the top and bottom frame are nonlinear.

A nonlinaerity is here observed for the top and bottom frame.
The top frame is therefore linearised in the angles $0^\circ$, $90^\circ$ and $180^\circ$ by calculating the Jacobian matrix.
As for the bottom frame's nonlinearity, it is shown that the influence caused by the position of the top frame is negligible. This is shown by a step response, in which the top frame is set to the aforementioned angles.



Based on the modelled system, a PI-controller has been tuned with the gains $K_p$ and $T_i$. It has then further been tuned by hand, in order to achieve the performance specifications to the best extend possible.

The PI-controller has been implemented on the Tiva, by deriving the difference equation using Tustin's method.
Furthermore step responses of the physical system have been logged. This is done to compare the physical and mathematical systems.
In the comparison it is seen that, the mathematical model have almost no overshoot, opposed to the actual system which capable of reaching an overshoot of 38\%.

Since the step responses are too dissimilar, it is concluded that the model itself is too inaccurate to be applied directly as a tool to tune the system. 

\paragraph{Digital System Design}
On the FPGA, decoders are implemented in order to measure the position of the system.
This is achieved by decoding the signals from the quadrature encoders of the pan-tilt system.
The resolution of the position measurement is $\frac{1}{3}^\circ$.
Based on the quadrature decoder, a unit on the FPGA for calculating the velocity has been implemented.

Furthermore the FPGA acts as the motor controller, this is achieved by an implementing the aforementioned PWM module in section \ref{sec:pwm}.

The PWM signal's frequency has been calculated to be $1003 \si{\,Hz}$, as this allow for the motor to draw full current, and therefore utilize its potential torque.

The FPGA can communicate with the Tiva by the use of a SPI protocol.

The implemented protocol follows the master/slave principle, in which the Tiva acts as the master and the FPGA as slave.

\paragraph{Embedded Programming}
A real time operating system, FreeRTOS, has been implemented on the Tiva. FreeRTOS has been chosen as this provides a scheduler, and thus making multiple task run in apparent concurrency possible.
The PI-controllers are implemented on the Tiva as tasks having the highest priority. This is to ensure the timings are consistent.


\end{document}
