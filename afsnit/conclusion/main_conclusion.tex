\documentclass[../../main]{subfiles}
\begin{document}
The control engineering: \\

FreeRTOS:\\
A real operating system, FreeRTOS, has been implemented on the Tiva. FreeRTOS has been chosen as it can emulate tasks running in parallel on the micrcontroller. \\
The controller is implemented on the Tiva as a task with the highest possible priority to make sure it has the best timing possible for the system.
\\

Encoder:\\
The controller needs a position in order to calculate the PWM values for the pan-tilt system. For this purpose quadrature encoders, mounted on the motors, are used.
A quadrature decoder program has been designed and written for the FPGA to decode the encoders' signals into a position.
It can determine the position with a $\frac{1}{3}$ degrees precision and direction, in which the system is rotating. A velocity has been implmented, which uses the position to calculate the velocity.
\\

PWM:\\
The PWM signal's frequency has been calculated to be $1003$Hz, as this allows the motor to draw full current and utilize all available torque. \\

SPI:
The SPI protocol is made as a shift register protocol


\begin{itemize}
  \item PI controller choice - should we have picked a PID/Modern control?
  \item control engineering:
  \begin{itemize}
    \item issue with nonlinaer modeling - how did we solve this?
    \item top and bottom frrame - two "individual" systems - are. theoretically fulfilling our requirements
    \item reality vs theoretical top and bottom frame - plus the finished handtuned values for Ki and Kp
  \end{itemize}
  \item design choices FPGA:
    \begin{itemize}
      \item Encoder - position counting with 1080 ticks or 1/3 degrees precision
      \item SPI - Protocol used
      \item PWM
      \item Homing?
    \end{itemize}
  \item embedded:
  \begin{itemize}
    \item Pi micrcontroller
    \item FreeRTOS
  \end{itemize}


\end{itemize}


\end{document}
