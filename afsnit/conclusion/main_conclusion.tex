\documentclass[../../main]{subfiles}
\begin{document}
The control engineering theory:
??
FreeRTOS:

The controller is implemented on the Tiva as a task with the highest possible priority to make sure it has the best timing possible for the system. It has been chosen to use FreeRTOS as the Tiva's operating system, because it can handle more than one task as well as offering preemptive scheduling.
Encoder:

The controller needs a position in order to calculate the PWM values for the pan-tilt system. For this purpose Quadrature Encoders, mounted on the motors, are used.
As the FPGA handles all direct contact with the pan-tilt system, a Quadrature Decoder program has been designed and written to decode the encoders' signals into a position.
It can detect the position with a $\frac{1}{3}$ degrees precision and direction as well as the velocity of which the system is rotating.

\begin{itemize}
  \item what are the performance requirements. - men det har vi allerede besvaret i system requirements?
  \item PI controller choice - should we have picked a PID/Modern control?
  \item issue with nonlinaer modeling - how did we solve this?
  \item top and bottom frrame - two "idividual" systems - are. theoretically fulfilling our requirements
  \item reality vs theoretical top and bottom frame - plus the finished handtuned values for Ki and Kp
  \item design choices software0: - important as it is a main point in the problem statement
    \begin{itemize}
      \item Encoder - position counting with 1080 ticks or 1/3 degrees precision
      \item SPI - Protocol used - did it do the trick
      \item PWM
      \item Homing?
    \end{itemize}
  \item implementation:
    \begin{itemize}
      \item Pi implementation - diffrence equation plus anti windup.
      \item FreeRTOS?
      \item FPGA - not part of the problem statement - only the design phrase
    \end{itemize}

\end{itemize}


\end{document}
