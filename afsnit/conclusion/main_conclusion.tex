\documentclass[../../main]{subfiles}
\begin{document}

In this section it is concluded to what extend the problems in the problem statements is solved.

\paragraph{Control engineering}

A mathematical model of the pan-tilt system has been made. \todo{It is noticed that in this model, both the top and bottom frame are nonlinear.}

A nonlinaerity is here observed for the top and bottom frame.
The top frame's nonlinaerity is approximated by linearization with the Jacobian matrix in angles $0^\circ$, $90^\circ$ and $180^\circ$.
\todo{The top frame is therefore linearized in the angles (insert angels) by calculating the Jacobian matrix. }
\todo{As for the bottom frame's non-linearity, it is shown that the influence caused by the position of the top frame is negligible. This is shown by a step response, in which the top frame is set to the aforementioned angles.}
It is  here determined that the bottom frame is dependent on the top frame, but the top frames effect is negligible.



Based on the modelled system, \todo{a PI-controller has been tuned with the gains $K_p$ and $T_i$. It has then further been tuned by hand, in order to achieve the performance specifications to the best extend possible. } a PI-controller has be with $K_p$ and $T_i$ gains have been tuned and then tuned by hand to comply with the performance specifications to the best extend possible.

\todo{The PI-controller has been implemented on the tiva, by deriving the difference equation using Tustin's method. }
This controller is implemented on the using tustin's method Tiva.
\todo{Futhermore step responses of the acutal system have been logged. (This is in order to achieve a comparison of the ...)  This is to compare the physical and mathematical systems. }
Step responses of the actual system have been logged and compared with the expected step responses from the mathematical model.
\todo{In the comparison it is seen that, the mathematical model have almost no overshoot, opposed to the actual system which capable of reaching an overshoot of 38\% }
The step responses based on the mathematical model have no overshoot compared with the actual system.

\todo{Since the step responses are too dissimilar, it is concluded that the model itself is too inaccurate to be applied directly as a tool to tune the system.   }
Based on the results it is concluded that the mathematical model is not accurate enough to apply directly.\\

\paragraph{Digital System Design}
On the FPGA, decoders are implemented in order to measure the positon of the system.
This is achieved by decoding the signals from the quadrature encoders of the pan-tilt system.

The FPGA can determine the position with a \todo{The resolution of the position and direction measurement is  } $\frac{1}{3}$ degrees of precision, and direction, in which the system is rotating.
\todo{Based on the quadrature decoder, a velocity }
A velocity has been implemented, which can be calculated from the position.\\

. This is done with a implemented PWM module.\\
The PWM signal's frequency has been calculated to be $1003 \si{\,Hz}$, as this allows the motor to draw full current and utilize all available torque. \\
The SPI protocol is a shift register protocol with a register size varying from 8 bits to a maximum of 16 bits. The protocol follows the master/slave princple, with the Tiva as the master and FPGA as slave. This protocol has been implemented on the FPGA by using a SPI module.

\paragraph{Embedded Programming}
A real time operating system, FreeRTOS, has been implemented on the Tiva. FreeRTOS has been chosen as it can emulate tasks running in parallel. \\
The controller is implemented on the Tiva as a task with the highest priority to make sure it has the most consistent timing possible for the controller.

\begin{itemize}
  \item PI controller choice
  \item control engineering:
  \begin{itemize}
    \item issue with nonlinaer modeling - how did we solve this?
    \item top and bottom frrame - two "individual" systems - are. theoretically fulfilling our requirements
    \item reality vs theoretical top and bottom frame - plus the finished handtuned values for Ki and Kp
  \end{itemize}
  \item design choices FPGA:
    \begin{itemize}
      \item Encoder - position counting with 1080 ticks or 1/3 degrees precision
      \item SPI - Protocol used
      \item PWM
      \item Homing?
    \end{itemize}
  \item embedded:
  \begin{itemize}
    \item Pi micrcontroller
    \item FreeRTOS
  \end{itemize}


\end{itemize}


\end{document}
