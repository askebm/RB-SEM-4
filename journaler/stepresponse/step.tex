\documentclass[../../main]{subfiles}
\begin{document}
\subsection{Stepresponse of system}%
\label{sub:stepresponse_of_system}

\paragraph{Purpose}%
\label{par:purpose}

Deterine stepresponse of system, monitoring current, position, and velocity for use in tuning system parameters.

\paragraph{Apperateur}%
\label{par:apperateur}
\begin{itemize}
	\item DSO-X 2024A (Osciloscope)
	\item Powersupply 12\si{V} and 5\si{V}
	\item 0.1\si{\Omega}5\% power resistor
\end{itemize}


\paragraph{Approach}%
\label{par:approach}
\todo{Add circuit}
The steprepsonse is performed at 200 pwm wich is $200\frac{12\si{V}}{255} = 9.4\si{V}$.\\
Data is sampled at 2.5\si{kHz} to properly filter out pwm signal of 1\si{kHz}.


\begin{figure}[H]
        \centering
				\def\svgwidth{0.47\columnwidth}
				\subfloat[Stepresponse of bottomframe with top frame at 0 degrees.]{%
				\includesvg[\main/journaler/]{step_bot}}
				\def\svgwidth{0.47\columnwidth}
				\subfloat[Stepresponse of top frame, with start at 0 degrees.]{%
				\includesvg[\main/journaler/]{step_top}}
				\caption{Stepresponse results of both frames.}
				\label{fig:jour_step_bot}
\end{figure}





\end{document}
